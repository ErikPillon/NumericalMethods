% !TeX spellcheck = en_US
\section{Model the Electrical Problem with the Finite Elements Method}
\subsection{Study Domain and Mesh}
Since the domian of interest of the electric problem is larger than the thermal problem, we have to slightly modify our algorithm. We then have the algorithm in Listing \ref{points_electric}, where we have put in input of our function \texttt{mesh electric} the number of points in which we want to discretize our domain.

\lstinputlisting[label={points_electric},caption={Points}]{Matlab_Code/points_electric.m}


\lstinputlisting[label={mesh_electric},caption={Build relationship for local point and overall point}]{Matlab_Code/mesh_electric.m}

\subsection{Galerkin’s Formulation of the Electrical Equation}
The projection of the partial differential equation on a basis element is
\[\iiint_{\omega} \alpha_i \nabla\cdot(-\sigma\nabla U)\diff\Omega=0, \quad \forall i \]
and then the \textbf{weak formulation for the electric part} becomes
\begin{equation}
\iiint_{\omega}\sigma\alpha_i\cdot\nabla U\diff \Omega=0,\quad \forall i.
\end{equation}
The boundary conditions given can be modelized in the following fashion
\[\begin{cases}
U = 0, &\text{on the top of the crucible,}\\
U = U_0, &\text{on the bottom of the crucible,}\\
J\cdot n=0, &\text{on all the other surfaces.}
\end{cases} \]
As before, the elementary element of surface and volume are given by the canonical change of variables, from cartesian to cylindrical, i.e., \begin{align}\label{change2}
(x,y,z)&\to(r,\theta,z)\\
(x,y,z)&\mapsto(rcos(\theta),rsin(\theta),z)\nonumber
\end{align}
The change of variable given by Eq.\ref{change2} gives as determinant of the Jacobian matrix the element $ r $, in such a way that the integration must be performed changing $ \diff x\diff y\diff z $ into $ r\diff r\diff\theta\diff z $.

The second change of variables will be 
\begin{align}\label{change3}
(r,\theta,z)&\to(\xi,\eta)\\
(r,\theta,z)&\mapsto(\xi(r,z),\eta(r,z)\nonumber
\end{align}
and we will have $ dxdydz=det(Jac)d\xi d\eta $.
\subsection{Algorithm of the Finite Element Formulation of the Electrical Equation}
\begin{mdframed}
	The algorithm goes as follows:
	\begin{itemize}
		\item Discretize the domain and build the mesh.
		\item For each element $ e $ of the domain:
		\begin{itemize}
			\item Construct the Jacobian matrix $ Jac $ with the functions $ \alpha_i $
			\item Evaluate the Jacobian $ \det(Jac) $
			\item Evaluate all the 16 terms $ \nabla\alpha_i\nabla\alpha_j $ with the help of the Gauss quadrature formula.
			\item Update the global matrix $ A $ with the function \texttt{Update}
		\end{itemize}
		\item For each element $ f $ of the upper boundary:
		\begin{itemize}
			\item Construct the Jacobian $ Jac $ (that is already a scalar value) with the functions $ \alpha_1 $ and $ \alpha_2 $
			\item Evaluate all the 4 terms $ \alpha_i\alpha_j $ with the help of the 1D Gauss quadrature formula.
			\item Update the global matrix $ B $ with the function \texttt{Update} modified for the boundary elements
		\end{itemize}
		\item Solve the linear problem $ (A_{ij}+B_{ij})T_i = Q_i $
	\end{itemize}
\end{mdframed}


In the program we make use of the following functions:
\begin{itemize}
	\item \texttt{jacobian(M, Elem, e)} where \texttt{M} is the coordinates matrix and \texttt{Elem} is the mesh matrix, while \texttt{e} is the global element taken into consideration.
	\item \texttt{alpha(xi,eta)} where \texttt{xi,eta} are the local coordinates.
	\item \texttt{update} for updating the global matrix with the 16 values obtained for each element $ e $. 
	\item The functions \texttt{points\_electric} and \texttt{mesh} as already stated in Section \ref{sec:4.1} 
	\item The function \texttt{Update\_electric} for putting the values of the local matrix of each element $ e $ (resp. $ f $) in the global matrix $ A $ (resp. $ B $).
\end{itemize}



\subsection{Programming and calculation}
\subsubsection{Construct the program: main program and functions}
\lstinputlisting[label={main_electric},caption={main program}]{Matlab_Code/main2.m}

Where we used the function \texttt{jacobian} and \texttt{alpha} listed below.

\lstinputlisting[label={jacobian_electric},caption={Function for the creation of the jacobian matrix for each element}]{Matlab_Code/jacobian.m}

\lstinputlisting[label={alpha},caption={Function for the creation of the alpha functions and the derivatives of the latter}]{Matlab_Code/alpha.m}

\lstinputlisting[label={update},caption={Function for update the global matrix with the values obtained for the local element $ e $}]{Matlab_Code/Update.m}

\lstinputlisting[label={dIntegral},caption={Function for evaluating the double integral$ \iint\nabla\alpha_i\nabla\alpha_j $}]{Matlab_Code/double_integral.m}


